% -*- root: ../main.tex -*-
\chapter{Retrospettiva}

\section{Svolgimento}
Gli sprint sono stati portati avanti nel seguente modo:
    \paragraph{Sprint Planning}
        Pianificazione a inizio sprint degli obiettivi, tempistiche e responsabilità nel periodo dello sprint corrente. Diviso in due parti:
        \begin{itemize}
        \item\textbf{parte 1} 
            Viene raffinato e rivisto il product backlog, viene effettuata la scelta dello sprint goal (what).
        \item\textbf{parte 2}
            Si decidono gli item e viene raffinato come implementarli (how). Effettuato con solo il team senza la figura del product owner
        \end{itemize}
    \paragraph{[Iterativo] Daily scrum} Breve meeting svolto giornalmente. Viene utilizzato per gli aggiornamenti sull'andamento del progetto, senza scendere nei dettagli implementativi.
    \paragraph{[Occasionale] Pair Programming } Utilizzato per risolvere problemi che causano il blocco di un componente del team per parecchio tempo su una issue.
    \paragraph{Meeting finale}
        Riflessioni e considerazioni finali sullo spint passato. Suggerimenti per migliorare il prossimo. Diviso in tre parti: 
        \begin{itemize}
        \item\textbf{Product backlog refinement} aggiunta di dettagli e riordino del product backlog
        \item\textbf{Sprint review} è stato ispezionato l'incremento, il Minimum Viable Product o di risultati sul processo. Discernere cosa è stato fatto e cosa no
        \item\textbf{Retrospettiva} Considerazioni sul team stesso e sui miglioramenti per il prossimo sprint. 
        \end{itemize}

% NOTA: queste parti qua sotto sono prese dalla cartella "process", all'interno dei vari sprint, solo la "sprint review" di ognuno
\section{Sprint 1}
    \section{Sprint 1 Review}
%only this part is present also in the main report
Durante questo primo sprint abbiamo completato l'organizzazione di massima.
\paragraph{Deliverables} 
I deliverables per questo sprint sono stati i seguenti:
\begin{itemize}
    \item Intervista con il cliente corredata da domande
    \item Ubiquitous Language
    \item Setup Organizzazione GitHub
    \item Setup Report, con relativa repository e CI
\end{itemize}

\section{Sprint 2}
    \section{Sprint 2 Review}
%only this part is present also in the main report
Durante questo sprint abbiamo realizzato l'analisi dei requisiti del sistema e ideata l'architettura generale del sistema, dalla quale sono sorti alcuni dubbi che sono stati poi discussi e chiariti con l'esperto del dominio. Sono state realizzare due versioni dei mockup dell'applicazione, le quali hanno ricevuto entrambe giudizi positivi, ma andranno successivamente unite e raffinate per soddisfare al meglio i requisiti del committente.
\paragraph{Deliverables} 
I deliverables per questo sprint sono stati i seguenti:
\begin{itemize}
    \item Mockup
    \item Diagramma dei casi d'uso
    \item Diagramma delle classi
\end{itemize}

\section{Sprint 3}
    \section{Sprint 3 Review}
%only this part is present also in the main report
Abbiamo studiato/implementato...
\paragraph{Deliverables} 
obiettivi raggiunti
\begin{itemize}
    \item uno
    \item due
    \item tre
\end{itemize}


