% -*- root: ../main.tex -*-

% Esporre brevemente le considerazioni conclusive sul progetto presentato, indicando anche i possibili sviluppi futuri.
% 1500 - 3000 battute

\chapter{Conclusioni}\label{conclusioni}
    Al termine dello sviluppo dell'applicazione, il prodotto rispecchia le richieste fatte inizialmente dall'esperto del dominio. In particolare, sono stati rispettati tutti i requisiti funzionali e sono anche stati aggiunti due requisiti funzionali opzionali, quali la modalità di gioco con sfida a tempo e le statistiche relative al tempo medio di risposta. Per poter rendere l'applicazione utilizzabile su larga scala bisognerebbe adottare delle accortezze particolari di seguito indicate.

    \section{Sviluppi Futuri}
    Tutti i requisiti funzionali opzionali precedentemente citati (capitolo \ref{chap:req_non_funzionali}), tranne quelli già sopra indicati come implementati, sono sicuramente i primi aspetti da inserire in uno sviluppo futuro, dal momento che sono stati segnalati dall'esperto del dominio durante l'analisi iniziale.
    In più, dato che il progetto sviluppato è un'applicazione desktop, non è difficile immaginarsi ulteriori funzionalità aggiuntive quali:
    
    \begin{itemize}
        \item salvataggio di tutti i dati del giocatore e delle partite in una architettura cloud o server based: si può creare una classifica generale con tutti gli studenti utilizzatori dell'applicazione, basata su quante partite hanno effettuato e quante risposte corrette hanno dato sul totale dei quiz presenti;
    
        \item sfida tra più studenti: in questa modalità di gioco più utenti possono sfidarsi utilizzando ciascuno l'applicativo sul proprio dispositivo;
        
        \item condivisione dei quiz inseriti da un utente: chiunque può avere a disposizione ulteriori corsi e/o quiz sui quali esercitarsi, cercando di coprire il più possibile gli argomenti presenti negli esami;
        
        \item segnalazione da parte dei giocatori di quiz da correggere, perché incompleti o inesatti.
    \end{itemize}
    
    

    
    