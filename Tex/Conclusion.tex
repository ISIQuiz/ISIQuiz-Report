% -*- root: ../main.tex -*-

% Esporre brevemente le considerazioni conclusive sul progetto presentato, indicando anche i possibili sviluppi futuri.
% 1500 - 3000 battute

\chapter{Conclusioni}\label{conclusioni}
    Complessivamente sono state implementate tutte le user story richieste dal domain expert e anche uno dei requisiti opzionali riguardante la modalità di gioco a tempo.

    \section{Sviluppi Futuri}
    Tutti i requisiti funzionali opzionali sopra citati (capitolo \ref{chap:req_non_funzionali}) sono sicuramente i primi aspetti da inserire, dal momento che sono stati indicati dall'esperto del dominio durante l'analisi iniziale.
    In più, dato che il progetto sviluppato è un'applicazione desktop, non è difficile immaginarsi ulteriori funzionalità aggiuntive:
    
    \begin{itemize}
        \item Salvataggio di tutti i dati del giocatore e delle partite in una architettura cloud o server based: si può creare una classifica generale con tutti gli studenti, basata su quante partite hanno effettuato e quante risposte corrette hanno dato sul totale dei quiz;
    
        \item Sfida tra più studenti: in questa modalità di gioco più utenti possono sfidarsi utilizzando l'applicativo sui propri dispositivi;
        
        \item Condivisione dei quiz inseriti da un utente: chiunque può avere a disposizione ulteriori corsi e/o quiz sui quali esercitarsi, cercando di coprire il più possibile gli argomenti presenti negli esami;
        
        \item Segnalazione da parte dei giocatori di quiz da correggere, perché incompleti o inesatti.
    \end{itemize}
    
    

    
    