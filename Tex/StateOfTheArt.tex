% -*- root: ../main.tex -*-

% Riassumere le soluzioni presenti in letteratura inerenti al problema in esame. Per ciascuna, discutere le principali diversità o affinità rispetto al progetto presentato. Nel caso non siano presenti soluzioni direttamente comparabili a quella presentata descrivere comunque le principali tecniche note per affrontare la tematica trattata. Le soluzioni esposte devono essere corredate degli opportuni riferimenti bibliografici. Nel caso si tratti di soluzioni già operative sul mercato, devono essere indicate le fonti (online) dove poter accedere al servizio o approfondirne i contenuti.
% 3000 - 6000 battute

\chapter{Stato dell'Arte}
 Di seguito si riporta una breve discussione sullo stato dell'arte, sui competitors e sulle soluzioni simili a quella proposta.
 
 Allo stato attuale, infatti, sono molte le varianti di giochi a quiz disponibili. Primariamente la motivazione risiede nel fatto che del format è stato fatto largo uso fin dall'antichità \cite{quizgame}, e con l'avvento della televisione ha avuto anche una seconda ondata di interesse \cite{gameshow}. Le varianti più di rilievo mantengono sempre il concetto di rispondere ad una domanda o scegliendo tra più risposte quella corretta \cite{whowantstobeamillionaire}, o con degli aiuti di diversa natura, o con un tempo limite \cite{jeopardy}. 
 Dal punto di vista istruttivo, piattaforme per giochi a quiz sono state sviluppate anche con ruolo didattico \cite{quizlet}, ma le soluzioni trovate sono orientate alla creazione dei quiz da parte degli insegnanti per gli studenti.
 
 La nostra idea si differenzia, in parte, per: 
\begin{itemize}
    \item la natura accademica orientata a corsi specifici di un determinato corso di laurea e laurea magistrale;
    \item l'utente target, in quanto ne usufruisce lo studente e non il docente.
\end{itemize}
 
L'intenzione con la quale si vuole portare avanti il progetto è quella di produrre un prodotto superiore a quelli disponibili nel mercato per una specifica necessità riscontrata, non apportare innovazione tecnologica.