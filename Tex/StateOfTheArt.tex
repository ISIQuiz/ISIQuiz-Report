% -*- root: ../main.tex -*-

% Riassumere le soluzioni presenti in letteratura inerenti al problema in esame. Per ciascuna, discutere le principali diversità o affinità rispetto al progetto presentato. Nel caso non siano presenti soluzioni direttamente comparabili a quella presentata descrivere comunque le principali tecniche note per affrontare la tematica trattata. Le soluzioni esposte devono essere corredate degli opportuni riferimenti bibliografici. Nel caso si tratti di soluzioni già operative sul mercato, devono essere indicate le fonti (online) dove poter accedere al servizio o approfondirne i contenuti.
% 3000 - 6000 battute

\chapter{Stato dell'Arte}
 ... non so se volgiamo farlo, ma forse un due righe non guastano
 (Rimuovere capitolo e unire "la nostra idea si differenzia per" nel capitolo 1)
 
Il topic scelto ha parecchie varianti:

\begin{itemize}
    \item Uno
    \item Due ...
\end{itemize}
 
La nostra idea si differenzia per: 

\begin{itemize}
    \item Natura di Accademica
    \item Ripasso possibile ...
    \item Commistione con il concetto di FlashCard
\end{itemize}