\chapter{Guida Utente}
Una volta aperta l'applicazione, viene visualizzato il menu iniziale:

\paragraph{Gioca}
Seleziona almeno un corso per poter iniziare una partita con quiz riguardanti i corsi indicati.
Successivamente, scegli la modalità di gioco:
\begin{itemize}
    \item partita standard: 10 quiz, ciascuno da rispondere in massimo 15 secondi; ogni quiz ha 4 possibili risposte, 1 sola risposta è corretta; scoprirai la correttezza o meno della risposta data dopo ogni quesito
    \item partita blitz: rispondi a più quiz possibili in 2 minuti (120 secondi); ogni quiz ha 4 possibili risposte, 1 sola risposta è corretta; scoprirai la correttezza o meno delle risposte date solo al termine della partita
    \item partita personalizzata: scegli il numero di quiz a cui rispondere e il tempo massimo (in secondi) in cui puoi rispondere a ciascuno; ogni quiz ha 4 possibili risposte, 1 sola risposta è corretta
\end{itemize}

Al termine di una partita c'è il riepilogo con:
\begin{itemize}
    \item il numero totale di domande risposte correttamente
    \item il numero totale di punti guadagnati
    \item per ogni quiz nel gioco:
    \begin{itemize}
        \item corso in cui è inserito
        \item risposta data
        \item risposta corretta, se quella data è errata
    \end{itemize}
\end{itemize}

\paragraph{Statistiche}
Visualizza le statistiche riguardanti ai tuoi ripassi:
\begin{itemize}
    \item relative ad un corso selezionato
    \item relative ad un quiz selezionato tra quelli del corso indicato
    \item globali
\end{itemize}
 
Per ciascuna tipologia di statistica ci sono:
\begin{itemize}
    \item punti acquisiti
    \item quiz risposti
    \item risposte corrette
    \item precisione (\%)
    \item tempo medio di risposta (sec)
\end{itemize}

\paragraph{Impostazioni}
Per poter apportare modifiche nelle impostazioni generali:
\begin{itemize}
    \item importa quiz: importa un file JSON dal file system contenente nuovi quiz
    \item esporta quiz: esporta il file JSON contenente i corsi con i relativi quiz
    \item modifica corso
    \item modifica quiz
    \item aggiungi corso: inserisci un nuovo corso
    \item aggiungi quiz: inserisci nuovi quiz ad un determinato corso, così da poterti esercitare su più argomenti
\end{itemize}

\paragraph{Esci}
Per uscire dall'applicazione