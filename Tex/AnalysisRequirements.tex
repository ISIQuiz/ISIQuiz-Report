% -*- root: ../main.tex -*-

% In questa sezione esporre brevemente i requisiti a cui il sistema proposto deve rispondere, concentrando l'attenzione sugli aspetti più rilevanti e facendo eventualmente uso di opportuni diagrammi di alto livello.
% 8000 - 10000 battute
% Attenzione in particolare ai requirement non funzionali: 1) non siano troppo vaghi altrimenti sono inverificabili, e quindi praticamente inutili; 2) se il sistema è distribuito, è inevitable dire cosa vi aspettate in termini di di robustezza a cambiamenti/guasti (quali?, come?), e scalabilità (in quale dimensione? fino a che punto?).

\chapter{Analisi dei Requisiti}
Come abbiamo trovato i \textbf{requisiti}
	\section{Requisiti di Business}
	...
	
	\section{Requisiti Utente}
    Uno user dell'applicazione dovrà quindi essere in grado di:
    \begin{itemize}
        \item ripassare: visualizzare domande e scegliere tra le possibili risposte
        \item scegliere tra varie modalità di gioco
        \item visualizzare un riepilogo a fine partita
        \item visualizzare le proprie statistiche su tutte le partite effettuate
        \item importare ed esportare nuove domande e risposte indicando la materia a cui si riferiscono
    \end{itemize}
 
	Si possono inserire qui i mockup
	    
	\section{Requisiti Funzionali} 
        \subsection{Obbligatori}
        All’interno di ogni partita, vengono presentate all’utente diverse domande di natura accademica, ognuna delle quali viene accompagnata da un numero predeterminato di risposte possibili, alcune corrette ed altre sbagliate. L’utente dovrà individuare le risposte corrette in un tempo massimo per non fallire la domanda. 

        \begin{itemize}
            \item \textbf{Modalità di gioco generale}: domande a scelta multipla su tutti i corsi disponibili
            
            \item \textbf{Modalità di gioco materie a scelta}: domande a scelta multipla su più materie scelte dal giocatore
            
            \item \textbf{Modalità di gioco materia specifica}: domande a scelta multipla su una sola materia scelta dal giocatore
            
            \item \textbf{Interfaccia grafica} (CLI e poi JavaFX): visualizzazione menu principale, visualizzazione della domanda e di 4 possibili risposte da scegliere durante il gioco, visualizzazione dei risultati post partita
            
            \item \textbf{Punteggio finale} del quiz appena effettuato: al termine della partita verrà visualizzato il punteggio ottenuto dal giocatore in base al numero di risposte corrette o errate date nella partita conclusa
            
            \item \textbf{Visualizzazione delle statistiche personali}
            
            \item \textbf{Più risposte giuste} e sbagliate per ogni domanda in modo da avere una rotazione tra le possibili scelte
            
            \item Aggiunta di \textbf{nuove domande e risposte da parte dell’utente}
        \end{itemize}  

    
        \subsection{Opzionali}
        \begin{itemize}
            \item \textbf{Sfida a tempo}: rispondere a più domande possibili in un intervallo di tempo limitato
            \item \textbf{Aiuti su richiesta}: all’utente vengono messi a disposizione un numero specifico di aiuti ad ogni partita con i quali semplificare la scelta della risposta
            \item \textbf{Revisione solo delle risposte sbagliate}: l’utente può scegliere una modalità di gioco in cui ripassare esclusivamente le domande precedentemente sbagliate
            \item \textbf{Medaglie-achievement} quando si completano delle sfide (es. più punti di un certo limite, più giornate continuativamente, aver risposto bene a tutte le domande di una materia)
        \end{itemize}
	\section{Requisiti non Funzionali}
        \begin{itemize}
            \item Interfaccia grafica intuitiva e che fornisca feedback coerenti all'utente per comunicargli lo stato delle sue azioni (ad esempio indicando con colore verde e checkmark una risposta qualora essa sia corretta)
            \item Interfaccia grafica accessibile agli utenti daltonici
            \item Interfaccia grafica reattiva: alle azioni di un utente devono corrispondere degli aggiornamenti dell'interfaccia stessa in tempi ragionevoli per non rovinare la user experience
            \item Realizzare un sistema modulare che permetta estensioni future (ad esempio uno sviluppo distribuito che permetta di realizzare funzionalità di gioco multiplayer)
            \item Il sistema deve essere funzionante in diversi sistemi operativi nei quali è installata una appropriata Java Virtual Machine (il requisito minimo considera il funzionamento su sistemi Windows, Linux e MacOS)
            \item Il sistema deve essere fault-tolerant, cioè deve continuare a funzionare in modo affidabile e deve essere privo di falle che permettano al giocatore di barare.
        
        \end{itemize}

	\section{Requisiti Implementativi}
	...
	