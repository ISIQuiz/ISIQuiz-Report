% -*- root: ../main.tex -*-

% Esporre i principali problemi affrontati durante l'effettiva realizzazione delle componenti hardware/software e illustrare le soluzioni implementative adottate. Se l'elaborato ha previsto l'utilizzo di tecnologie già disponibili sul mercato, discuterne brevemente le caratteristiche e motivarne l'adozione rispetto ad altre soluzioni assimilabili. NOTA: in questa sezione devono essere riportate esclusivamente le porzioni di codice ritenute particolarmente significative. 
% 10000 - 21000 battute
%per ogni studente, una sotto-sezione descrittiva di cosa fatto/co-fatto e con chi, e descrizione di aspetti implementativi importanti non già presenti nel design
\chapter{Implementazione}\label{chap:impl}
Nell'implementazione non ci sono aspetti particolari da segnalare in quanto lo sviluppo del codice ha seguito piuttosto fedelmente quanto deciso nel design del sistema. Si riportano quindi, per ogni membro del team, i principali componenti implementati ed i ruoli assunti durante il progetto.

\section{Gambaletta Daniele}
    All'interno del progetto ho ricoperto il ruolo di \textbf{domain expert}, quindi sono stato intervistato all'inizio per capire i requisiti dell'applicativo, ho controllato i \textit{mockup} e li ho fatti correggere in base ai miei gusti personali, controllando che venissero rispettate le mie indicazioni durante lo sviluppo.
    Per quanto riguarda l'implementazione, nel \textit{model} mi sono occupato principalmente delle statistiche e di parte dei corsi. Inoltre ho sviluppato le \textit{utils} necessarie alla persistenza dei quiz e delle statistiche, gestendo sia la parte di \textit{parsing} dei dati in formato JSON che quella di salvataggio/caricamento dei \textit{files}. 
    
    Model:
    \begin{itemize}
        \item PlayerStats
        \item CourseInStats
        \item QuizInStats - in collaborazione con Teodorani
        \item SavedCourse - in collaborazione con Teodorani
        \item Course
        \item CourseIdentifier - in collaborazione con Teodorani
        \item Quiz - in collaborazione con Teodorani
        \item GameStage - in collaborazione con Omiccioli, Lirussi e Teodorani
        \item Session - in collaborazione con Teodorani
    \end{itemize}

    View:
    \begin{itemize}
        \item Settings Menu View - in collaborazione con Teodorani, Lirussi
        \item Statistics Menu View - in collaborazione con Teodorani
    \end{itemize}
    
    FXML:
    \begin{itemize}
        \item Statistics
    \end{itemize}
    
    Controller:
    \begin{itemize}
        \item AppController - in collaborazione con Omiccioli
        \item StandardGameController - in collaborazione con Omiccioli, Lirussi e Teodorani
        \item BlitzGameController - in collaborazione con Omiccioli, Lirussi
        \item StatisticsMenuController - in collaborazione con Omiccioli
    \end{itemize}

    Utils:
    \begin{itemize}
        \item JsonParser con CourseJsonParser e StatsJsonParser
        \item FileHandler
        \item DataStorageHandler 
        \item ImportHandler
        \item ExportHandler
    \end{itemize}
    
    
\section{Lirussi Igor}
    Per la durata del progetto, ricoprendo il ruolo aggiuntivo di \textbf{scrum-master}, ho svolto il compito di gestire il processo di sviluppo (capitolo \ref{chap:dev-process}) dal punto di vista della metodologia Agile, rispettando le sue varie fasi e utilizzando il più possibile il framework Scrum. Ricoprendo questo ruolo, la responsabilità è stata quella di portare il focus dello sviluppo sul processo, più che sul risultato. Mi sono occupato di predisporre l'organizzazione con le relative repository e i tool da utilizzare, collegando ad essi la Continuous Integration (capitolo \ref{chap:CI}) della reportistica e del software, affinché i cambiamenti vengano versionati e ci sia \textit{accountabiliy} su di essi. Inoltre, è stato aggiunto il setup della Quality Assurance (capitolo \ref{chap:QA}) con test, documentazione e coverage, garantendo così la massima attenzione sul metodo di sviluppo. Infine, ad essa sono stati affiancati i bot, i siti generati ed i relativi artefatti. Durante l'implementazione del software, ho organizzato le fasi di sviluppo affinché utilizzassimo correttamente i principi scelti, mantenendo il product backlog (capitolo \ref{chap:product-backlog}), gli sprint e i task con i relativi branch (capitolo \ref{chap:git-flow}).
    
    Riguardo alla parte di implementazione, il mio focus è stato sull'aggiunta di quiz (Requisiti \ref{chap:req_funzionali}.\ref{us:aggiunta-quiz}), di corsi (Requisiti \ref{chap:req_funzionali}.\ref{us:aggiunta-corso}) e la modifica di quiz (Requisiti \ref{chap:req_funzionali}.\ref{us:modifica-quiz}) e di corsi (Requisiti \ref{chap:req_funzionali}.\ref{us:modifica-corso}). Questo mi ha portato ad occuparmi nel model dell'implementazione di quiz con domande, e in \textit{controller} e \textit{view} della relativa logica e visualizzazione. Un'altra area di particolar focus è stata il riepilogo finale della partita, come da (Requisiti \ref{chap:req_funzionali}.\ref{us:revisione-quiz}).  Nel \textit{model} mi sono occupato del Game Stage e della Review in esso contenuta, per utilizzarli nei \textit{controller} relativi, i quali comunicano con le interfacce scelte (grafiche e da linea di comando).

    Controller:
    \begin{itemize}
        \item Add course controller - in collaborazione con Omiccioli 
        \item Add quiz controller - in collaborazione con Omiccioli
        \item Edit course controller - in collaborazione con Omiccioli
        \item Edit quiz controller - in collaborazione con Omiccioli
        \item Review controller - in collaborazione con Omiccioli
    \end{itemize}    
    Model:
    \begin{itemize}
        \item Quiz - in collaborazione con Teodorani e Gambaletta
        \item Answer - in collaborazione con Teodorani
        \item Game Stage - in collaborazione con Omiccioli, Teodorani e Gambaletta
        \item Review 
    \end{itemize}
    Utils:
    \begin{itemize}
        \item Timer - in collaborazione con Omiccioli
    \end{itemize}
    View:
    \begin{itemize}
        \item Add course view  - in collaborazione con Teodorani
        \item Add quiz view - in collaborazione con Teodorani
        \item Edit course view - in collaborazione con Teodorani
        \item Edit quiz view - in collaborazione con Teodorani
        \item Review view
        \item Settings Menu View - in collaborazione con Gambaletta e Teodorani
    \end{itemize}
    FXML:
    \begin{itemize}
        \item Add course menu - in collaborazione con Omiccioli
        \item Add quiz menu - in collaborazione con Omiccioli
        \item Edit course menu - in collaborazione con Omiccioli
        \item Edit quiz menu - in collaborazione con Omiccioli
        \item Review menu - in collaborazione con Omiccioli e Teodorani
    \end{itemize}
    
\section{Omiccioli Riccardo}
   In qualità di \textbf{product owner} del progetto ho: presieduto i vari meeting svolti dal team di sviluppo, identificato i requisiti principali (raffinati poi con il team) dell'applicativo richiesto e organizzato il lavoro del team. Il mio lavoro si è principalmente concentrato nel design e realizzazione dell'architettura generale dell'applicazione implementando, in particolare, la parte relativa ai \textit{controller}, alle \textit{view} e alla comunicazione tra essi.
   
    FXML:
    \begin{itemize}
        \item Add course menu - in collaborazione con Lirussi
        \item Add quiz menu - in collaborazione con Lirussi
        \item Blitz game
        \item Custom menu
        \item Default menu
        \item Edit course menu - in collaborazione con Lirussi
        \item Edit quiz menu - in collaborazione con Lirussi
        \item Main menu
        \item Review menu - in collaborazione con Lirussi
        \item Select menu - in collaborazione con Teodorani
        \item Standard game - in collaborazione con Teodorani
    \end{itemize}
    Controller:
    \begin{itemize}
        \item Package actions
        \item App controller
        \item Blitz game controller
        \item Controller
        \item Custom menu controller
        \item Game controller - in collaborazione con Teodorani
        \item Main menu controller
        \item Standard game controller - in collaborazione con Teodorani
        \item Select menu controller - in collaborazione con Teodorani
    \end{itemize}
    Model:
    \begin{itemize}
        \item Game settings
        \item Game stage - in collaborazione con Gambaletta, Lirussi e Teodorani
        \item Timer - in collaborazione con Lirussi
    \end{itemize}
    View:
    \begin{itemize}
        \item Blitz game menu
        \item Blitz view
        \item Custom menu
        \item Default menu
        \item Main menu - in collaborazione con Teodorani
        \item Main menu view - in collaborazione con Teodorani
        \item Select menu - in collaborazione con Teodorani
        \item Select menu view - in collaborazione con Teodorani
        \item Standard game menu
        \item Package updates
        \item View - in collaborazione con Teodorani
    \end{itemize}
    Altro:
    \begin{itemize}
        \item Main
    \end{itemize}
    
\section{Teodorani Cecilia}
    Insieme ad Omiccioli, ho seguito lo sviluppo generale dell'architettura, realizzando alcuni \textit{controller}, \textit{view} e componenti del \textit{model}. Di quest'ultimo mi sono anche occupata di standardizzare tutto il codice al termine dell'implementazione del progetto. In più, mi sono assicurata che il \textit{model} fosse adeguatamente testato, sviluppando test molto dettagliati. Infine, per quanto riguarda la GUI, ho messo a punto tecniche per la user experience e l'accessibilità. Per la prima ho adottato tecniche standard, quali ad esempio il font e la disposizione dei vari componenti nella pagina, mentre per la seconda ho utilizzato dei tool, per garantire che soddisfacessero le \textit{WCAG (Web Content Accessibility Guidelines)}. In particolare, per il colore primario del sito e il colore del testo sovrastante, si è utilizzato \href{https://m2.material.io/resources/color/#!/?view.left=1&view.right=1&primary.color=8a171a&secondary.color=DEDEDE}{Material Design - Color Tool} e \href{https://webaim.org/resources/contrastchecker/?fcolor=DEDEDE&bcolor=8A171A}{WebAIM}.
    
    FXML:
    \begin{itemize}
        \item Review Menu - in collaborazione con Lirussi
        \item Select Menu - in collaborazione con Omiccioli
        \item Standard Game - in collaborazione con Omiccioli
    \end{itemize}
    Controller:
    \begin{itemize}
        \item Select Menu Controller - in collaborazione con Omiccioli
        \item Game Controller - in collaborazione con Omiccioli
        \item Standard Game Controller - in collaborazione con Gambaletta, Omiccioli e Lirussi
    \end{itemize}
    Model:
    \begin{itemize}
        \item Answer - in collaborazione con Lirussi
        \item Game Stage - in collaborazione con Gambaletta, Lirussi e Omiccioli
        \item Course Identifier - in collaborazione con Gambaletta
        \item Quiz - in collaborazione con Gambaletta, Lirussi
        \item SavedCourse - in collaborazione con Gambaletta
        \item Session - in collaborazione con Gambaletta
        \item QuizInStats - in collaborazione con Gambaletta
    \end{itemize}
    View (realizzata la struttura di tutti gli update in \textit{nomeFile}MenuView sotto riportati, mentre dei controller grafici \textit{nomeFile}Menu la struttura generale e la navigazione tra le varie pagine):
    \begin{itemize}
        \item Add Course Menu - in collaborazione con Lirussi
        \item Add Course Menu View - in collaborazione con Lirussi
        \item Add Quiz Menu - in collaborazione con Lirussi
        \item Add Quiz Menu - in collaborazione con Lirussi
        \item Custom Menu
        \item Custom Menu View - in collaborazione con Omiccioli
        \item Main menu - in collaborazione con Omiccioli
        \item Main Menu View - in collaborazione con Omiccioli
        \item Select Menu - in collaborazione con Omiccioli
        \item Select Menu View - in collaborazione con Omiccioli
        \item Settings Menu View - in collaborazione con Gambaletta, Lirussi
        \item Standard Game Menu - in collaborazione con Omiccioli
        \item Statistics Menu View - in collaborazione con Gambaletta
        \item View - in collaborazione con Omiccioli
    \end{itemize}

    