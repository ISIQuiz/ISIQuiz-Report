% -*- root: ../main.tex -*-

% Esporre i principali problemi affrontati durante l'effettiva realizzazione delle componenti hardware/software e illustrare le soluzioni implementative adottate. Se l'elaborato ha previsto l'utilizzo di tecnologie già disponibili sul mercato, discuterne brevemente le caratteristiche e motivarne l'adozione rispetto ad altre soluzioni assimilabili. NOTA: in questa sezione devono essere riportate esclusivamente le porzioni di codice ritenute particolarmente significative. 
% 10000 - 21000 battute
%per ogni studente, una sotto-sezione descrittiva di cosa fatto/co-fatto e con chi, e descrizione di aspetti implementativi importanti non già presenti nel design
\chapter{Implementazione}
...
    \section{Gambaletta Daniele}
    All'interno del progetto ho ricoperto il ruolo di \textbf{domain expert}, quindi sono stato intervistato all'inizio per capire i requisiti dell'applicativo, ho controllato i mockup e li ho fatti correggere in base ai miei gusti personali, ho controllato che venissero rispettate le mie indicazioni durante lo sviluppo e dato feedback finali (presenti nel capitolo \ref{conclusioni}) una volta che è stata rilasciata la versione definitiva.
    
    \section{Lirussi Igor}
    Ho svolto il compito di gestire il progetto dal punto di vista della metodologia Scrum facendo lo \textbf{scrum master}. Ho controllato durante tutte le fasi del progetto che utilizzassimo correttamente i principi scelti del lavoro agile, controllando e sistemando le issue, gli sprint e il product backlog.
    
    \section{Omiccioli Riccardo}
   In qualità di \textbf{product owner} del progetto ho: presieduto i vari meeting svolti dal team di sviluppo, identificato i requisiti principali (raffinati poi con il team) dell'applicativo richiesto e organizzato il lavoro del team.
    
    \section{Teodorani Cecilia}

    Per scegliere il colore primario del sito e il colore del testo sovrastante, si è utilizzato \href{https://m2.material.io/resources/color/#!/?view.left=1&view.right=1&primary.color=8a171a&secondary.color=DEDEDE}{Material Design - Color Tool}, così che avessero il giusto contrasto e, per controllare che veramente soddisfacessero le \textit{WCAG (Web Content Accessibility Guidelines)}, è stato utilizzato \href{https://webaim.org/resources/contrastchecker/?fcolor=DEDEDE&bcolor=8A171A}{WebAIM}.