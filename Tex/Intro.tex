% -*- root: ../main.tex -*-

% Esporre l'obiettivo del progetto dandone una visione complessiva. Devono essere illustrate le caratteristiche salienti del progetto; deve essere chiara la distinzione tra le tecnologie usate/assemblate durante lo svolgimento dell'elaborato e il contributo tecnologico/scientifico e effettivamente apportato dal gruppo.
% 3000 - 6000 battute

\chapter{Introduzione}
ISIQuiz è un progetto che nasce con l'intento di sviluppare un \textbf{gioco a quiz} che permetta ad uno studente di ripassare il contenuto dei corsi di \textbf{Ingegneria e Scienze Informatiche} attraverso domande a scelta multipla che coprono una o più materie di suo interesse. Lo scopo di \textbf{ISIQuiz} è quindi quello di gamificare il ripasso pre-esame attraverso uno strumento stimolante che permetta agli studenti di tracciare i propri progressi.
%img
\begin{figure}[H]
    \label{fig:Logo}
    \centering
    \includegraphics[width=0.8\textwidth]{Extra/ISIQuizLogoLineTransparent.png}
    \caption{Il logo del progetto}
\end{figure}

\section{Overview}
L'idea è nata dall'osservazione degli incontri tra studenti nelle giornate antecedenti ad una prova orale o scritta. Per preparasi al meglio sulle nozioni teoriche, nel gruppo si crea una dinamica simile a docente-studente, dove a turno ognuno fa domande di carattere generale su alcuni argomenti alle quali gli altri presenti cercano di rispondere. In questo modo, lo studente che risponde alle domande è in grado di ripassare e, con un numero sufficiente di quesiti, coprire gli argomenti dell'intera materia. Questa modalità di ripasso in gruppo favorisce una copertura migliore del programma del corso rispetto ad uno studio individuale, nel quale alcuni argomenti potrebbero essere involontariamente trascurati.
Lo scopo di questo progetto è quindi quello di cercare di ricreare le dinamiche appena indicate nel caso in cui non sia possibile un incontro di gruppo tra gli studenti.