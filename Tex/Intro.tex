% -*- root: ../main.tex -*-

% Esporre l'obiettivo del progetto dandone una visione complessiva. Devono essere illustrate le caratteristiche salienti del progetto; deve essere chiara la distinzione tra le tecnologie usate/assemblate durante lo svolgimento dell'elaborato e il contributo tecnologico/scientifico e effettivamente apportato dal gruppo.
% 3000 - 6000 battute

\chapter{Introduzione}
ISIQuiz è un progetto che nasce con l'intento di sviluppare un \textbf{gioco a quiz}, che permetta ad uno studente di ripassare il contenuto dei corsi di \textbf{Ingegneria e Scienze Informatiche} attraverso domande a scelta multipla che coprono una o più materie di suo interesse. Lo scopo di \textbf{ISIQuiz} è quindi quello di gamificare il ripasso pre-esame attraverso uno strumento stimolante che permetta agli studenti di tracciare i propri progressi e allenarsi sulle domande che per lui risultano più critiche.

\section{Overview}
L'idea è nata dall'osservazione dei meeting di studenti coetanei nelle giornate antecedenti un prova orale o scritta che contenga della teoria. In tale periodo è chiaro che l'"ansia da esame" raggiunge il suo picco, per la scarsità di tempo per lo studio e per l'esame imminente. Qualora ci siano delle nozioni non pratiche, infatti, nel gruppo si crea una dinamica simil docente-studente, dove a turno ognuno fa domande ad altri di carattere generale sugli argomenti che meglio padroneggia. Curiosamente, il ruolo preferito sembra essere quello di colui a cui vengono rivolte le domande. Ciò è inconsciamente volto a mitigare la paura di aver "mancato" qualche topic nello studio, o di non ricordarselo a sufficienza. Infatti, essere testati su concetti generali da altri studenti che li padroneggiano meglio rassicura notevolmente lo studente in questione. Inoltre, la dinamica "a gruppo" garantisce una copertura adeguata di tutte le nozioni più importanti. L'idea è quindi quella di creare un applicazione che garantisca una esperienza simile, quando la situazione sopracitata non è possibile. 
%img
\begin{figure}[H]
    \caption{Il \textbf{logo} in-line}
    \label{fig:Logo}
    \centering
    \includegraphics[width=0.8\textwidth]{Extra/ISIQuizLogoLineTransparent.png}
\end{figure}

\section{Problemi}
I problemi che hanno portato alla necessità di creare il software quindi sono: 
    \begin{itemize}
        \item Incertezza sulla memorizzazione degli argomenti studiati 
        \item Incertezza sulla copertura di tutti gli argomenti dell'esame
        \item Poco tempo per il ripasso o ulteriore studio
    \end{itemize}
 

\section{Obiettivi}
L'obiettivo è quello di mitigare l'ansia da esame fornendo uno strumento valido che soddisfi la necessità, condivisa da ogni studente, di fare un "recap" antecedente ad un esame.
Infine, sarebbe opportuno che questo processo sia reso il più positivo e piacevole possibile.

