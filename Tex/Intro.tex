% -*- root: ../main.tex -*-

% Esporre l'obiettivo del progetto dandone una visione complessiva. Devono essere illustrate le caratteristiche salienti del progetto; deve essere chiara la distinzione tra le tecnologie usate/assemblate durante lo svolgimento dell'elaborato e il contributo tecnologico/scientifico e effettivamente apportato dal gruppo.
% 3000 - 6000 battute

\chapter{Introduzione}
ISIQuiz è un progetto che nasce con l'intento di sviluppare un \textbf{gioco a quiz}, che permetta ad uno studente di ripassare il contenuto dei corsi di \textbf{Ingegneria e Scienze Informatiche} attraverso domande a scelta multipla che coprono una o più materie di suo interesse. Lo scopo di \textbf{ISIQuiz} è quindi quello di gamificare il ripasso pre-esame attraverso uno strumento stimoltante che permetta allo studenti di tracciare i propri progressi e allenarsi sulle domande che per lui risultano più critiche.

\section{Overview}

%img
\begin{figure}[H]
    \caption{Il \textbf{logo} in-line}
    \label{fig:Logo}
    \centering
    \includegraphics[width=0.8\textwidth]{Extra/ISIQuizLogoLineTransparent.png}
\end{figure}

\section{Problemi}
I problemi che hanno portato alla necessità di svolgere il sw:
\begin{itemize}
    \item Uno
    \item Due
    \item Tre
    \item Quattro
\end{itemize}
 




\section{Obiettivi}
L'obiettivo è quello di ... 




