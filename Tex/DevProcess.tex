% -*- root: ../main.tex -*-
\chapter{Processo di Sviluppo}

\section{Domain Driven Design}
Il processo di sviluppo ...
    \subsection{Aspetti principali}

    
    Concetti chiave ...:
    
        \begin{itemize}
        \item \textbf{Uno}: ...
        \item \textbf{Due}: ...
        \item \textbf{Tre}: ...
       
        
    \end{itemize}

 

\section{Metodologia di Sviluppo}
Per sviluppare il progetto è stata scelto il framework Scrum. Questo per permettere ... 
    \subsection{Scrum}
    Sprint, Sprint Planning,  product-backlog, daily-Scrum, "Definition of Done", "refinement" del backlog, 
    

\section{Gestione di Progetto}
In questa sezione verrà spiegato come il progetto...
    \paragraph{Gantt Chart} 
    ... 
    
    \paragraph{Licensing} 
    ...
    
    \paragraph{Versioning}
    ...
    
    \paragraph{GitHub Projects }
    Per le Board di Scrum è stato utilizzato GitHub Projects...
    
    \paragraph{Telegram}
    scelta telegram...
        \subparagraph{Bot} 
        teleram bot...
    
    \paragraph{Discord}
    Discord... 

\section{Continuous Integration e Automatizzazione}
\label{chap:CI}
CI e Automatizzazione
    \subsection{Relazione}
        relazione su overleaf e relativa CI
        
        \subparagraph{Tool utilizzati}
        \begin{itemize}
            \item Overleaf
            \item GitHub Actions
            \item Telegram Bots
            \item Pandoc
            \item GitHub Pages
        \end{itemize}

    \subsection{Progetto}
        CI progetto, Test, Coverage, Documentazione, Quality assurance, bla bla
        
        \subparagraph{Tool utilizzati}
        \begin{itemize}
            \item Sbt
            \item ScalaTest
            \item GitHub Actions
            \item GitHub Pages
            \item Telegram Bots
            \item Dependabot
            \item Sonarcloud/Codacy/Codefactor
        \end{itemize}








