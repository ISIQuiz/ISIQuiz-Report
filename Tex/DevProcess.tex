% -*- root: ../main.tex -*-
%modalità di divisione in itinere dei task, meeting/interazioni pianificate, modalità di revisione in itinere dei task, scelta degli strumenti di test/build/continuous integration
\chapter{Processo di Sviluppo}

\section{Domain Driven Design}
Il processo di sviluppo ...
    \subsection{Aspetti principali}

    
    Concetti chiave ...:
    
        \begin{itemize}
        \item \textbf{Uno}: ...
        \item \textbf{Due}: ...
        \item \textbf{Tre}: ...
       
        
    \end{itemize}

 

\section{Metodologia di Sviluppo}
Per sviluppare il progetto è stata scelto il framework Scrum. Questo per permettere ... 
    \subsection{Scrum}
    Secondo il framework "Scrum" il lavoro va diviso in più sprint seguendo un approccio iterativo. Il team mantiene due tipi di backlog, il product backlog e lo sprint backlog.. Sprint Planning... Product-backlog... Daily-Scrum... Definition of Done ... "refinement" del backlog... 
    \linebreak\linebreak
    \textbf{Definition of done:} Un item del product backlog si ritiene concluso quando tutti i task che lo compongono sono stati completati, il codice implementato è stato adeguatamente testato con esito positivo e la relativa documentazione è stata scritta.
    \linebreak\linebreak
    \textbf{Scrum poker:} Per stimare il livello di effort necessario per il completamento di ogni task è stata utilizzata la tecnica che viene chiamata \textbf{scrum poker}. Questa tecnica consiste nella lettura e discussione di un task e degli aspetti che lo riguardano e nella scelta, da parte di ogni membro del team, di un valore di effort stimato scegliendo tra 1, 2, 3, 5, 8, 13, 20 il numero che ritiene più adeguato a rappresentare la complessità del task tenendo conto ad esempio del tempo ritenuto necessario per lo sviluppo o la complessità stimata del task stesso. A seguito della scelta di un numero da parte di ogni membro vengono rivelati i numeri scelti e si cerca di raggiungere consenso sulla scelta del numero finale, eventualmente argomentando la propria decisione. La scelta del set di numeri da assegnare è tale da avere volutamente ampi intervalli tra i numeri per ridurre conflitti e raggiungere con maggiore semplicità una situazione di consenso.

\section{Gestione di Progetto}
In questa sezione verrà spiegato come il progetto...
    \paragraph{Gantt Chart} 
    ... 
    
    \paragraph{Licensing} 
    ...
    
    \paragraph{Versioning}
    ...
    
    \paragraph{GitHub-Projects}
    Per le Board di Scrum è stato utilizzato GitHub-Projects...
    
    \paragraph{Telegram}
    scelta Telegram...
        \subparagraph{Bot} 
        Telegram bot...
    
    \paragraph{Discord}
    Discord... 

\section{Continuous Integration e Automatizzazione}
\label{chap:CI}
CI e Automatizzazione
    \subsection{Relazione}
        relazione su overleaf e relativa CI
        
        \subparagraph{Tool utilizzati}
        \begin{itemize}
            \item Overleaf
            \item GitHub Actions
            \item Telegram Bots
            \item Pandoc
            \item GitHub Pages
        \end{itemize}

    \subsection{Progetto}
        CI progetto, Test, Coverage, Documentazione, Quality assurance, bla bla
        
        \subparagraph{Tool utilizzati}
        \begin{itemize}
            \item Sbt
            \item ScalaTest
            \item GitHub Actions
            \item GitHub Pages
            \item Telegram Bots
            \item Dependabot
            \item Sonarcloud/Codacy/Codefactor
        \end{itemize}








